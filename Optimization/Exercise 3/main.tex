\documentclass[a4paper,12pt]{report}

\usepackage{ucs}
\usepackage[utf8x]{inputenc} % Input encoding for Greek characters
\usepackage[greek,english]{babel} % Language support

\newcommand{\en}{\selectlanguage{english}}
\newcommand{\gr}{\selectlanguage{greek}}

% \usepackage{algorithm2e}
% \usepackage{algorithm}
% \usepackage{algorithmic}
\usepackage{enumitem}
\usepackage{amssymb}
\usepackage{float}
\usepackage{amsmath}
\usepackage{graphicx} % For including images
\usepackage{titlesec} % Custom title formatting
\usepackage{fancyhdr} % For custom headers and footers
\usepackage{geometry} % For adjusting page margins

% Adjust the page margins to make content wider
\geometry{top=2.5cm, bottom=2.5cm, left=2.5cm, right=2.5cm}

% Redefine chapter formatting to make it smaller
\titleformat{\chapter}[display]
    {\normalfont\LARGE\bfseries} % Smaller size and bold for chapter heading
    {\chaptername\ \thechapter} % Chapter number format
    {15pt} % Space between chapter number and title
    {\bfseries} % Smaller size and bold for chapter title
\begin{document}

\begin{titlepage}
    \centering
    \vspace*{-3cm}
    % University logo
    \includegraphics[width=1\textwidth]{auth_logo.png} % Replace with your actual logo file

    % University name in Greek
    \textbf{\gr ΑΡΙΣΤΟΤΕΛΕΙΟ ΠΑΝΕΠΙΣΤΗΜΙΟ ΘΕΣΣΑΛΟΝΙΚΗΣ}
    \vspace{2cm}

    % Document title and subtitle in Greek
    \LARGE\textbf{\gr Τεχνικές Βελτιστοποίησης Αναφορά} \\
    \Large\normalfont{\gr Εργασία 3} \\
    \vspace{4cm}

    \gr
    \large
    \textbf{Διακολουκάς Δημήτριος} \\
    \textbf{AEM 10642}
    \vspace{2.5cm}

    \en
    \textit{Email: ddiakolou@ece.auth.gr}
\end{titlepage}

\gr
\tableofcontents

\chapter{\gr Εισαγωγή και σχεδιασμός συνάρτησης}
Η τρίτη εργασία εστιάζει στη χρήση της \textbf{Μεθόδου Μέγιστης Καθόδου με Προβολή} για τη βελτιστοποίηση μιας συνάρτησης δύο μεταβλητών. Η συγκεκριμένη συνάρτηση που θα μελετηθεί είναι η \(f(x) = \frac{1}{3}x_1^2 + 3x_2^2\), όπου \(x = [x_1, x_2]^T\). Η εργασία περιλαμβάνει διάφορες παραλλαγές της μεθόδου, με στόχο τη σύγκριση της απόδοσης της σε διαφορετικά σενάρια βελτιστοποίησης, λαμβάνοντας υπόψη και περιορισμούς.

\vspace{0.5cm}

\hspace{-0.6cm}Ο βασικός στόχος της εργασίας είναι η διερεύνηση:
\begin{itemize}
    \item Της σύγκλισης της \textbf{Μεθόδου Μέγιστης Καθόδου} υπό διαφορετικές τιμές για την παράμετρο \(\gamma_k\) και με συγκεκριμένο βαθμό ακρίβειας \(\varepsilon\).
    \item Της συμπεριφοράς της μεθόδου όταν εισάγονται περιορισμοί στις μεταβλητές \(x_1\) και \(x_2\), δηλαδή \(-10 \leq x_1 \leq 5 \) και \(-8 \leq x_2 \leq 12\).
    \item Της βελτίωσης της σύγκλισης μέσω αλλαγών στις παραμέτρους \(\gamma_k\) και \(s_k\), καθώς και της χρήσης προβολών.
\end{itemize}

\vspace{0.5cm}

\hspace{-0.6cm}Η εργασία οργανώνεται σε τέσσερα θέματα:
\begin{enumerate}
    \item Στο πρώτο θέμα, εφαρμόζεται η \textbf{Μέθοδος Μέγιστης Καθόδου} χωρίς περιορισμούς, εξετάζοντας διαφορετικές τιμές της παραμέτρου \(\gamma_k\) και την επίδρασή τους στη σύγκλιση.
    \item Στο δεύτερο θέμα, εξετάζεται η μέθοδος υπό περιορισμούς στις μεταβλητές \(x_1\) και \(x_2\), χρησιμοποιώντας τον συντελεστή προβολής \(s_k = 5\) και διαφορετικό αρχικό σημείο.
    \item Στο τρίτο θέμα, εισάγεται η βελτίωση της μεθόδου με \(s_k = 15\), και προτείνεται ένας πρακτικός τρόπος για την επιτάχυνση της σύγκλισης.
    \item Στο τέταρτο θέμα, μελετάται η συμπεριφορά της μεθόδου όταν χρησιμοποιούνται εξαιρετικά μικρές τιμές για το \(s_k\), με στόχο την ανάλυση της ακρίβειας του αλγορίθμου.
\end{enumerate}

\vspace{0.5cm}

\hspace{-0.6cm}Η ανάλυση κάθε θέματος συνοδεύεται από διαγράμματα, αποτελέσματα και κώδικα \(MATLAB\) ο οποίος επισυνάπτεται ξεχωριστά.

\vspace{0.5cm}

\hspace{-0.6cm}Παρακάτω επισυνάπτεται και μία γραφική αναπαράσταση της συνάρτησης στο Σχήμα 1.1 που μελετάται σε αυτή την εργασία και εμφανίζεται κατά την εκτέλεση του αρχείου \en \textbf{MyFunction.m}\gr.


\begin{figure}[ht!]
    \centering
    \includegraphics[width=1\textwidth]{func.png} 
    \caption{Γενική εικόνα της μορφής της \(f\).}
\end{figure}

\chapter{\en Steepest Descent \gr για πολλαπλά βήματα \(\gamma_k\)}
\section*{Ανάλυση του Αλγορίθμου (Θέμα 1)}

Ο παρακάτω κώδικας υλοποιεί τη \textbf{Μέθοδο Μέγιστης Καθόδου} για τη βελτιστοποίηση μιας συνάρτησης δύο μεταβλητών. Αναλύουμε τα μαθηματικά βήματα και τη λειτουργία του αλγορίθμου.

\subsection*{Περιγραφή της Μεθόδου}

Η \textbf{Μέθοδος Μέγιστης Καθόδου} βασίζεται στη χρήση της κλίσης (\(\nabla f(x)\)) μιας συνάρτησης \(f(x)\) για τον προσδιορισμό της διεύθυνσης στην οποία η συνάρτηση \(f(x)\) μειώνεται περισσότερο. Ο αλγόριθμος εκτελεί επαναληπτικά βήματα σύμφωνα με τον τύπο:
\[
    x_{k+1} = x_k - \gamma_k \nabla f(x_k),
\]
όπου:
\begin{itemize}
    \item \(x_k\): το τρέχον σημείο,
    \item \(\gamma_k > 0\): ο ρυθμός μάθησης ή το μήκος βήματος,
    \item \(\nabla f(x_k)\): η κλίση της συνάρτησης στο σημείο \(x_k\),
    \item \(d_k = -\nabla f(x_k)\): η διεύθυνση μέγιστης καθόδου.
\end{itemize}

\subsection*{Eπεξήγηση Υλοποίησης του Αλγορίθμου}

\begin{enumerate}
    \item \textbf{Αρχικοποίηση:} \(x_0 = \text{\en start\_point\gr}, k = 0\).
    \item \textbf{Υπολογισμός της Κλίσης:}
    \[
    \nabla f(x_k) = \begin{bmatrix} \frac{\partial f}{\partial x_1}(x_k) \\ \frac{\partial f}{\partial x_2}(x_k) \end{bmatrix}.
    \]
    \item \textbf{Έλεγχος Σύγκλισης:}
    Εάν το μέτρο της κλίσης \(\|\nabla f(x_k)\| = \sqrt{\left(\frac{\partial f}{\partial x_1}(x_k)\right)^2 + \left(\frac{\partial f}{\partial x_2}(x_k)\right)^2}\) είναι μικρότερο ή ίσο από \(\varepsilon\), τότε ο αλγόριθμος σταματά.
    \item \textbf{Υπολογισμός Διεύθυνσης Καθόδου:}
    \[
    d_k = -\nabla f(x_k).
    \]
    \item \textbf{Υπολογισμός Επόμενου Σημείου:}
    Το επόμενο σημείο \(x_{k+1}\) υπολογίζεται ως εξής:
    \[
    x_{1,k+1} = x_{1,k} - \gamma_k \frac{\partial f}{\partial x_1}(x_k),
    \]
    \[
    x_{2,k+1} = x_{2,k} - \gamma_k \frac{\partial f}{\partial x_2}(x_k).
    \]
    \item \textbf{Επανάληψη:} Ενημερώνεται το πλήθος των επαναλήψεων \(k \leftarrow k + 1\) και επαναλαμβάνονται τα βήματα μέχρι τη σύγκλιση.
\end{enumerate}

\subsection*{Περιορισμοί της Μεθόδου}

Η \textbf{Μέθοδος Μέγιστης Καθόδου} είναι εύκολη στην υλοποίηση και αποτελεσματική για κυρτές συναρτήσεις, αλλά παρουσιάζει ορισμένα μειονεκτήματα:
\begin{itemize}
    \item Η ταχύτητα σύγκλισης εξαρτάται έντονα από την επιλογή του \(\gamma_k\). Μικρές τιμές οδηγούν σε αργή σύγκλιση, ενώ μεγάλες τιμές μπορεί να προκαλέσουν αστάθεια.
    \item Μπορεί να παγιδευτεί σε τοπικά ελάχιστα όταν η συνάρτηση δεν είναι κυρτή.
\end{itemize}


\subsection*{Επίδειξη και Μαθηματική απόδειξη αποτελεσμάτων}

Έχοντας λοιπόν την γενικότερη ιδέα για την λειτουργικότητα της μεθόδου Μεγίστης Καθόδου παρακάτω επισυνάπτωνται και τα αποτελέσματα από την εφαρμογή της στο \(MATLAB\) για διάφορες τιμές \(\gamma_k\). Ο αντίστοιχος κώδικας βρίσκεται στο αρχείο \en \textbf{Part1.m}\gr. Σε κάθε περίπτωση θα παρατηρήσετε ότι επιλέχθηκε αρχικό σημείο το (1, 1). Στο Σχήμα 2.1 αναπαρίστανται τα \en Plots \gr για \(\gamma_k\) ίσο με 0.1 στο Σχήμε 2.2 για \(\gamma_k\) ίσο με 0.3 στο Σχήμα 2.3 για \(\gamma_k\) ίσο με 3 και στο σχήμα 2.4 για \(\gamma_k\) ίσο με 5.

\begin{figure}[ht!]
    \centering
    \includegraphics[width=1\textwidth]{fig0_1.png} 
    \caption{Αναπαράσταση για \(\gamma_k\) ίσο με 0.1.}
\end{figure}

\begin{figure}[ht!]
    \centering
    \includegraphics[width=1\textwidth]{fig0_3.png} 
    \caption{Αναπαράσταση για \(\gamma_k\) ίσο με 0.3.}
\end{figure}

\begin{figure}[ht!]
    \centering
    \includegraphics[width=1\textwidth]{fig3.png} 
    \caption{Αναπαράσταση για \(\gamma_k\) ίσο με 3.}
\end{figure}

\begin{figure}[ht!]
    \centering
    \includegraphics[width=1\textwidth]{fig5.png} 
    \caption{Αναπαράσταση για \(\gamma_k\) ίσο με 5.}
\end{figure}

\clearpage
\vspace{0.5cm}

\hspace{-0.6cm}Παρατηρώ ότι στο Σχήμα 2.1 ο συνολικός αριθμός επαναλήψεων για \(\gamma_k = 0.1\) ανέρχεται σε \(k = 61\). Έπίσης παρατηρώ ότι για \(\gamma = 0.1\), το μήκος βήματος είναι αρκετά μικρό, και ο αλγόριθμος συγκλίνει προς το συνολικό ελάχιστο σταθερά και αργά με την \(x_1\) να συγκλίνει με βραδύτερο ρυθμό από την \(x_2\). Επομένως, η μεταβλητή \(x_2\) φτάνει στο ελάχιστο (0, 0) πιο γρήγορα από τη μεταβλητή \(x_1\). Στο Σχήμα 2.2 φαίνεται πως ο συνολικός αριθμός επαναλήψεων για \(\gamma_k = 0.3\) ανέρχεται σε \(k = 29\). Επιπλέον παρατηρώ ταλαντώσεις του \(x_2\) σε αντίθεση με το \(x_1\) που δεν παρουσιάζει. Αυτό συμβαίνει διότι για \(\gamma_k = 0.3\) προκαλείται μεγάλο βήμα, με αποτέλεσμα ο \(x_2\) να μην συγκλίνει απευθείας, αλλά να υπερβαίνει και να επιστρέφει προς το ελάχιστο, προκαλώντας ταλάντωση. Τέλος, στις περιπτώσεις όπου το \(\gamma_k = 3\) ή \(\gamma = 5\) δεν παρατηρείται σύγκλιση του αλγορίθμου όπως άλλωστε φαίνεται ξεκάθαρα και στα σχήματα Σχήμα 2.3 και 2.4 αντίστοιχα.

\vspace{0.5cm}

\hspace{-0.6cm}Aκολοθεί απόδειξη των αποτελεσμάτων που είδαμε παραπάνω και προέκυψαν από τον \(MATLAB\) κώδικα για τις διάφορες τιμές \(\gamma_k\) που ακολούθησαν με μαθηματική ακρίβεια. Έτσι μπορούν να εξηγηθούν και τα αποτελέσματα που πήραμε για τιμές \(\gamma_k\) ίσο με 0.1, 0.3, 3 και 5 αντίστοιχα. Έτσι, λοιπόν έχω μία αξιόλογη λύση για τον λόγο που για κάθε \(\gamma_k > \frac{1}{3}\) δεν έχω σύγκλιση. Παρακάτω, λοιπόν επισυνάπτω την απόδειξη όπως άλλωστε φαίνεται στο Σχήμα 2.5. 

\begin{figure}[ht!]
    \centering
    \includegraphics[width=1\textwidth]{myimage.jpg} 
    \caption{Μαθηματική Απόδειξη.}
\end{figure}

\chapter{\grΜέθοδος \en Steepest Descent with Projection \gr}
\section*{Ανάλυση του Αλγορίθμου}
\subsection*{Περιγραφή της Μεθόδου}
Στα υποερωτήματα 2, 3 και 4 θα περιγραφεί η χρήση της Μεθόδου Μέγιστης Καθόδου με Προβολή για την ελαχιστοποίηση μιας συνάρτησης υπό συγκεκριμένους περιορισμούς. Εξετάζονται διάφορες τιμές των παραμέτρων \(\gamma_k\) και \(s_k\), καθώς και διαφορετικά αρχικά σημεία, για να αναλυθεί η σύγκλιση του αλγορίθμου, οι ιδιαιτερότητές του και να προταθούν πρακτικές βελτιώσεις. Όπως και στην απλή μέθοδο Μεγίστης Καθόδου έτσι και στην μέθοδο Μεγίστης Καθόδου με Προβολή θα ακολουθήσουμε την παρόμοια υλοποίηση στον αλγόριθμό μας με μικρές διαφοροποιήσεις. 

\subsection*{Eπεξήγηση Υλοποίησης του Αλγορίθμου}
\begin{enumerate}
    \item \textbf{Αρχικοποίηση:} Ορίζεται το αρχικό σημείο \(x_0 \in \mathbb{R}^2\) και ο αριθμός των επαναλήψεων \(k = 0\). Η αρχική τιμή της συνάρτησης \(f(x_0)\) υπολογίζεται ως:
    \[
    f(x_0) = \frac{1}{3}x_{1,0}^2 + 3x_{2,0}^2.
    \]

    \item \textbf{Υπολογισμός της Κλίσης:} Για κάθε επανάληψη \(k\), υπολογίζεται η κλίση της συνάρτησης \(f(x)\) στο σημείο \(x_k\):
    \[
    \nabla f(x_k) = \begin{bmatrix} \frac{\partial f}{\partial x_1}(x_k) \\ \frac{\partial f}{\partial x_2}(x_k) \end{bmatrix} = \begin{bmatrix} \frac{2}{3}x_{1,k} \\ 6x_{2,k} \end{bmatrix}.
    \]

    \item \textbf{Κριτήριο Τερματισμού:} Εξετάζεται εάν το μέτρο της κλίσης \(\|\nabla f(x_k)\|\) ικανοποιεί την προϋπόθεση σύγκλισης:
    \[
    \|\nabla f(x_k)\| = \sqrt{\left(\frac{2}{3}x_{1,k}\right)^2 + (6x_{2,k})^2} \leq \varepsilon.
    \]
    Εάν η παραπάνω συνθήκη ικανοποιηθεί ή \(k > 40000\), η διαδικασία τερματίζεται.

    \item \textbf{Υπολογισμός Διεύθυνσης Καθόδου:} Η κατεύθυνση καθόδου \(d_k\) είναι η αρνητική κλίση:
    \[
    d_k = -\nabla f(x_k).
    \]

    \item \textbf{Ενδιάμεσο Σημείο:} Υπολογίζεται το προσωρινό σημείο \(x_{\text{temp}}\) χρησιμοποιώντας το βήμα \(s_k\):
    \en
    \[
    x_{\text{temp},1} = x_{1,k} + s_k d_{1,k}, \quad x_{\text{temp},2} = x_{2,k} + s_k d_{2,k}.
    \]
    \gr
    \item \textbf{Προβολή στα Όρια:} Λαμβάνονται υπόψη οι περιορισμοί του χώρου:
        \[
        -10 \leq x_1 \leq 5
        \]
        \[
        -8 \leq x_2 \leq 12
        \]
    για τις συνιστώσες του \(x_{\text{\en temp\gr}}\):
    \en
    \[
    x_{\text{next},1} = \min(\max(x_{\text{temp},1}, x_{\text{min}}), x_{\text{max}}), 
    \]
    \[
    x_{\text{next},2} = \min(\max(x_{\text{temp},2}, y_{\text{min}}), y_{\text{max}}).
    \]
    \gr
    \item \textbf{Ενημέρωση Σημείου:} Το επόμενο σημείο \(x_{k+1}\) υπολογίζεται με βάση τη χαλάρωση:
    \en
    \[
    x_{1,k+1} = x_{1,k} + \gamma_k (x_{\text{next},1} - x_{1,k}),
    \]
    \[
    x_{2,k+1} = x_{2,k} + \gamma_k (x_{\text{next},2} - x_{2,k}).
    \]
    \gr
    \item \textbf{Επανάληψη:} Καταγράφονται οι τιμές των \(x_k\), \(y_k\) και της συνάρτησης \(f(x_k)\):
    \[
    f(x_k) = \frac{1}{3}x_{1,k}^2 + 3x_{2,k}^2.
    \]
    Το πλήθος των επαναλήψεων ενημερώνεται \(k \leftarrow k + 1\), και η διαδικασία επαναλαμβάνεται μέχρι να ικανοποιηθεί το κριτήριο σύγκλισης.
\end{enumerate}

\vspace{0.5cm}

\hspace{-0.6cm}Οι περιορισμοί και αδυναμίες και αυτής της μεθόδου είναι αντίστοιχοι με αυτούς που συζητήθηκαν στην απλή μέθοδο Μεγίστης Καθόδου παραπάνω. Ο αντίστοιχος αλγόριθμος βρίσκεται στο αρχείο \en \textbf{Part2.m}\gr.

\subsection*{Επίδειξη αποτελεσμάτων και συμπεράσματα συγκρίσεων}
\subsubsection*{Θέμα 2}
Στο Θέμα αυτό έχουμε \(s_k = 5\), \(\gamma_k = 0.5\) και σημείο εκκίνησης το \((5, -5)\) παρατηρούμε στα διαγράμματα παρακάτω ότι δεν υπάρχει καμία συγκλιση στις 40000 επαναλήψεις. Σε αυτό το ερώτημα ακολουθώντας την ίδια ανάλυση που κάναμε για το πρώτο θέμα εύκολα απορρέει το συμπέρασμα ότι θέλουμε να ισχύει \(s_k \cdot \gamma_k < \frac{1}{3}\). Ωστόσο με βάση το Σχήμα 3.1 φαίνεται πως \(s_k \cdot \gamma_k = 2.5 > \frac{1}{3}\). Άρα ο αλγόριθμος αποκλίνει. Όπως είδαμε σε ένα απο τα παραδείγματα και στο Θέμα 1 έτσι και εδώ έχουμε ταλαντώσεις που δεν καταλήγουν σε σύγκλιση. Το πρόβλημα προκαλείται από το μεγάλο βήμα που οδηγεί την προβολή \(x_{2k}\) εκτός των ορίων του επιτρεπτού συνόλου και σε αυτή την περίπτωση. Ωστόσο το \(x_{1k}\) φαίνεται να συγκλίνει στο 0 λογικό καθώς \(s_k \cdot \gamma_k = 2.5 < 3\) που ισχύει ως περιορισμός για το \(x_{1k}\) όπως αποδείξαμε και στο Θέμα 1 κατ' αντιστοιχία.

\begin{figure}[ht!]
    \centering
    \includegraphics[width=1\textwidth]{fig0_5_5.png} 
    \caption{Αναπαράσταση για \(\gamma_k\) ίσο με 0.5 και \(s_k\) ίσο με 5.}
\end{figure}

\subsubsection*{Θέμα 3}
Στο Θέμα αυτό έχουμε \(s_k = 15\), \(\gamma_k = 0.1\) και σημείο εκκίνησης το \((-5, 10)\) αυτό παρατηρούμε στα διαγράμματα παρακάτω ότι η αντικειμενική συνάρτηση συγκλίνει στις 1215 επαναλήψεις όπως φαίνεται και στο Σχήμα 3.2. Σε αυτό το ερώτημα ακολουθώντας την ίδια ανάλυση που κάναμε για το προηγούμενο παρατηρούμε ότι \(s_k \cdot \gamma_k = 1.5 > \frac{1}{3}\). Για αυτό το \(x_{2k}\) ταλαντώνεται και χρειάζεται πολλές επαναλήψεις για σύγκλιση στο 0. Επίσης μπορούμε να καταλάβουμε την σύγκλιση που έχει το \(x_{1k}\) στο 0 καθώς \(s_k \cdot \gamma_k = 1.5 < 3\). Συνεπώς, έχω πολύ αργή σύγκλιση.

\vspace{0.5cm}

\hspace{-0.6cm}Ένας άλλος τρόπος να πετύχουμε σύγκλιση αλλά με λιγότερες επαναλήψεις έκρινα ότι είναι μειώνωντας το \(s_k = 15\) σε \(s_k = 2.5\) διατηρώντας το \(\gamma_k = 0.1\). Σε αυτή την περίπτωση έχουμε \(s_k \cdot \gamma_k = 0.25 < \frac{1}{3}\) και παρατηρούμε σύγκλιση στις 32 επαναλήψεις όπως φαίνεται και στο Σχήμα 3.3 που συγκριτικά φαίνεται ως μία αρκετά μεγάλη βελτίωση.

\begin{figure}[ht!]
    \centering
    \includegraphics[width=1\textwidth]{fig0_1_15.png} 
    \caption{Αναπαράσταση για \(\gamma_k\) ίσο με 0.1 και \(s_k\) ίσο με 15.}
\end{figure}

\begin{figure}[ht!]
    \centering
    \includegraphics[width=1\textwidth]{fig0_1_2_5.png} 
    \caption{Αναπαράσταση για \(\gamma_k\) ίσο με 0.1 και \(s_k\) ίσο με 2.5.}
\end{figure}

\clearpage

\subsubsection*{Θέμα 4}
Στο Θέμα αυτό έχουμε \(s_k = 0.1\), \(\gamma_k = 0.2\) και σημείο εκκίνησης το \((8, -10)\) παρατηρούμε στα διαγράμματα παρακάτω ότι υπάρχει συγκλιση στις 448 επαναλήψεις. Σε αυτό το ερώτημα ακολουθώντας την ίδια ανάλυση που κάναμε και στα προηγούμενα εύκολα απορρέει το συμπέρασμα ότι θέλουμε να ισχύει \(s_k \cdot \gamma_k < \frac{1}{3}\) πράγμα που ισχύει αφού \(s_k \cdot \gamma_k = 0.02\) και τα αποτελέσματα αναπαρίστανται και στο Σχήμα 3.4. Ωστόσο οι μικρές τιμές σε \(s_k\) αλλά και \(\gamma_k\) μας δείχνουν ότι συγκλίνουμε αργά. 

\begin{figure}[ht!]
    \centering
    \includegraphics[width=1\textwidth]{fig0_2_0_1.png} 
    \caption{Αναπαράσταση για \(\gamma_k\) ίσο με 0.2 και \(s_k\) ίσο με 0.1.}
\end{figure}

\bibliographystyle{plain}
\begin{thebibliography}{1}
    \bibitem{rovithakis}
    Γεώργιος Α. Ροβιθάκης, \textit{Τεχνικές Βελτιστοποίησης}. Εκδόσεις ΤΖΙΟΛΑ.
\end{thebibliography}

\end{document}
