\documentclass[a4paper,12pt]{report}

\usepackage{ucs}
\usepackage[utf8x]{inputenc} % Input encoding for Greek characters
\usepackage[greek,english]{babel} % Language support

\newcommand{\en}{\selectlanguage{english}}
\newcommand{\gr}{\selectlanguage{greek}}

% \usepackage{algorithm2e}
% \usepackage{algorithm}
% \usepackage{algorithmic}
\usepackage{url}
\usepackage{enumitem}
\usepackage{amssymb}
\usepackage{float}
\usepackage{amsmath}
\usepackage{graphicx} % For including images
\usepackage{titlesec} % Custom title formatting
\usepackage{fancyhdr} % For custom headers and footers
\usepackage{geometry} % For adjusting page margins

% Adjust the page margins to make content wider
\geometry{top=2.5cm, bottom=2.5cm, left=2.5cm, right=2.5cm}

% Redefine chapter formatting to make it smaller
\titleformat{\chapter}[display]
    {\normalfont\LARGE\bfseries} % Smaller size and bold for chapter heading
    {\chaptername\ \thechapter} % Chapter number format
    {15pt} % Space between chapter number and title
    {\bfseries} % Smaller size and bold for chapter title
\begin{document}

\begin{titlepage}
    \centering
    \vspace*{-3cm}
    % University logo
    \includegraphics[width=1\textwidth]{auth_logo.png} % Replace with your actual logo file

    % University name in Greek
    \textbf{\gr ΑΡΙΣΤΟΤΕΛΕΙΟ ΠΑΝΕΠΙΣΤΗΜΙΟ ΘΕΣΣΑΛΟΝΙΚΗΣ}
    \vspace{2cm}

    % Document title and subtitle in Greek
    \LARGE\textbf{\gr Τεχνικές Βελτιστοποίησης Αναφορά} \\
    \Large\normalfont{FINAL PROJECT} \\
    \vspace{4cm}

    \gr
    \large
    \textbf{Διακολουκάς Δημήτριος} \\
    \textbf{AEM 10642}
    \vspace{2.5cm}

    \en
    \textit{Email: ddiakolou@ece.auth.gr}
\end{titlepage}

\gr
\tableofcontents

\chapter{\gr Εισαγωγή και επίδειξη προβλήματος}
Στην αναφορά αυτή θα αναλυθε το πρόβλημα της ελαχιστοποίησης μιας συνάρτησης πολλών μεταβλητών, με την εφαρμογή γενετικών αλγορίθμων. Το πρόβλημα βασίζεται σε ένα οδικό δίκτυο, όπου οι κόμβοι αναπαριστούν διασταυρώσεις και οι ακμές κυκλοφοριακές κατευθύνσεις. 
Συγκεκριμένα, κάθε ακμή του γράφου χαρακτηρίζεται από σταθερές και δυναμικές παραμέτρους που επηρεάζουν το χρόνο διέλευσης των οχημάτων. Στόχος είναι να βρεθεί ο αριθμός των οχημάτων που ελαχιστοποιεί το συνολικό χρόνο διέλευσης στο δίκτυο, διατηρώντας ισορροπία στη ροή οχημάτων ανά κόμβο.

\vspace{0.4cm}
\hspace{-0.6cm}Στην αναφορά αυτή εξετάζονται τα παρακάτω:
\begin{itemize}
    \item Δίνεται η μαθηματική διατύπωση του προβλήματος.
    \item Υλοποιείται ένας γενετικός αλγόριθμος σε περιβάλλον \texttt{\en Matlab\gr}.
    \item Εξετάζεται η ευαισθησία της λύσης σε μεταβολές της εισερχόμενης ροής οχημάτων.
\end{itemize}


\chapter{\gr Μαθηματική Διατύπωση του Προβλήματος}
Το πρόβλημα πραγματεύεται την ελαχιστοποίηση του συνολικού χρόνου διέλευσης σε ένα οδικό δίκτυο. Το δίκτυο απεικονίζεται στο Σχήμα~\ref{fig:network}, όπου οι κόμβοι αντιπροσωπεύουν διασταυρώσεις και οι ακμές κυκλοφοριακές κατευθύνσεις. 

\section*{\gr Περιγραφή του Προβλήματος}
Οι ακμές του δικτύου έχουν χαρακτηριστικά που επηρεάζουν τον χρόνο διέλευσης των οχημάτων:
\begin{itemize}
    \item $t_i$: ο σταθερός χρόνος που απαιτείται για την κίνηση στο δρόμο $i$ όταν η κίνηση είναι ασθενής.
    \item $x_i$: ο ρυθμός διέλευσης των οχημάτων στο δρόμο $i$.
    \item $c_i$: ο μέγιστος δυνατός ρυθμός διέλευσης οχημάτων στο δρόμο $i$.
    \item $a_i$: μία σταθερά που εξαρτάται από το δρόμο $i$.
\end{itemize}

\hspace{-0.6cm}Ο χρόνος κίνησης σε κάθε δρόμο $i$ δίνεται από τη σχέση:
\begin{equation}
    T_i(x_i) = t_i + a_i \frac{x_i}{1 - \frac{x_i}{c_i}},
\end{equation}
με:
\[
\lim_{x_i \to 0} T_i(x_i) = t_i, \quad \text{και} \quad \lim_{x_i \to c_i} T_i(x_i) = +\infty.
\]

\section*{\gr Στόχος}
Σκοπός είναι η ελαχιστοποίηση του συνολικού χρόνου διέλευσης $T$ στο δίκτυο:
\begin{equation}
    T = f(x_i) = \sum_{i=1}^{17} T_i(x_i),
\end{equation}
υπό τις εξής συνθήκες:
\begin{enumerate}
    \item Ο ρυθμός εισερχόμενων οχημάτων είναι ίσος με $V$.
    \item Η κατανομή των οχημάτων στους κόμβους είναι τέτοια ώστε όσα οχήματα εισέρχονται σε κάθε κόμβο, τόσα να εξέρχονται.
\end{enumerate}
Να σημειωθεί ότι στα παρακάτω ερωτήματα επιλέχθηκε η μεγιστοποίηση της $\frac{1}{f(x_i)}$ ως \en fitness function \gr για την ελαχιστοποίηση του $T$ στον γενετικό αλγόριθμο που υλοποιήθηκε.

\vspace{0.3cm}
\hspace{-0.6cm}Οι παράμετροι του προβλήματος δίνονται ως εξής:
\begin{itemize}
    \item Οι τιμές του $c_i$ για τις ακμές του δικτύου είναι:
    \[
    \begin{bmatrix}
        54.13 & 21.56 & 34.08 & 49.19 & 33.03 & 21.84 & 29.96 \\
        24.87 & 47.24 & 33.97 & 26.89 & 32.76 & 39.98 & 37.12 \\
        53.83 & 61.65 & 59.73
    \end{bmatrix}
    \]
    \item Οι τιμές του $a_i$ είναι:
    \[
    a_i = 
    \begin{cases} 
        1.25, & i = 1, \dots, 5, \\ 
        1.5, & i = 6, \dots, 10, \\ 
        1, & i = 11, \dots, 17.
    \end{cases}
    \]
    \item Ο συνολικός ρυθμός εισερχόμενων οχημάτων:
    \[
    V = 100.
    \]
    \item Τυχαία επιλογή σταθεράς \(t_i\):
    \[
    t_i = 3, \quad i = 1, \dots, 5.
    \]
\end{itemize}

\section*{\gr Σχήμα Δικτύου, Eξισώσεις και Περιορισμοί}
Το δίκτυο φαίνεται στο Σχήμα~\ref{fig:network}, όπου αναγράφονται οι τιμές $c_i$ για κάθε ακμή.
\begin{figure}[h]
    \centering
    \includegraphics[width=0.6\textwidth]{Graph.png} 
    \caption{\gr Το οδικό δίκτυο.}
    \label{fig:network}
\end{figure}

\vspace{0.2cm}

\hspace{-0.6cm}Όσον αφορά τις εξισώσεις επειδή ο ρυθμός εισόδου ισούται με τον ρυθμό εξόδου εύκολα απορρέει το συμπέρασμα ότι με βάση και τις ακμές $c_i$ και τα βέλη:
\begin{itemize}
    \item \(x_1 + x_2 + x_3 + x_4 = V\)
    \item \(x_1 = x_5 + x_6\)
    \item \(x_2 = x_7 + x_8\)
    \item \(x_4 = x_9 + x_{10}\)
    \item \(x_3 + x_8 + x_9 = x_{11} + x_{12} + x_{13}\)
    \item \(x_{13} + x_7 + x_6 = x_{14} + x_{15}\)
    \item \(x_{14} + x_5 = x_{16}\)
    \item \(x_{11} + x_{10} = x_{17}\)
    \item \(x_{17} + x_{12} + x_{15} + x_{16} = V\)
\end{itemize}

\hspace{-0.6cm}Έπιπλέον ζητώ να ισχύουν:
\begin{itemize}
    \item \(x_i \geq 0, \quad \forall i\)
    \item \(x_i - c_i \leq 0, \quad \forall i\)
\end{itemize}
Χρειάζομαι αυτούς τους δύο περιορισμούς ώστε να εξασφαλίζεται η φυσική λογική του προβλήματος, ότι οι δρόμοι δεν μπορούν να υποστηρίξουν ροή μεγαλύτερη από τη χωρητικότητά τους και ότι δεν υπάρχει αρνητική ροή οχημάτων.

\chapter{\gr Υλοποίηση Γενετικού Αλγορίθμου}
Η υλοποίηση του γενετικού αλγορίθμου πραγματοποιείται με στόχο την ελαχιστοποίηση του συνολικού χρόνου διέλευσης (μεγιστοποίηση της συνάρτησης καταλληλότητας που έχει οριστεί ως $\frac{1}{f(x_i)}$ στο δίκτυο (γράφο) υπό τους περιορισμούς που ορίζονται στο πρόβλημα και αναλύθηκαν παραπάνω στην Μαθηματική Διατύπωση. Παρακάτω περιγράφονται τα κύρια στάδια του αλγορίθμου.


\section*{\gr Δημιουργία Αρχικού Πληθυσμού \en (Initial Population)\gr}

Η αρχική φάση του αλγορίθμου περιλαμβάνει τη δημιουργία ενός αρχικού πληθυσμού, δηλαδή μιας συλλογής πιθανών λύσεων. Η κάθε λύση αναπαρίσταται από έναν πίνακα που περιέχει τις ροές \(x_i\) σε κάθε δρόμο του δικτύου. Οι ροές αυτές πρέπει να ικανοποιούν βασικούς περιορισμούς όπως υποδείχτηκαν και παραπάνω:
\begin{itemize}
    \item Κάθε ροή \(x_i\) είναι μη αρνητική (\(x_i \geq 0\)).
    \item Κάθε ροή \(x_i\) δεν υπερβαίνει τη χωρητικότητα του δρόμου (\(x_i \leq c_i\)).
    \item Οι ροές συνδέονται μέσω εξισώσεων ισορροπίας στους κόμβους.
\end{itemize}
Η τυχαιότητα στην αρχική κατανομή των ροών διασφαλίζει τη διαφορετικότητα στον πληθυσμό, κάτι που είναι απαραίτητο για την αποφυγή τοπικών ελαχίστων.

\section*{\gr Συνάρτηση Καταλληλότητας \en (Fitness Function)\gr}

Η καταλληλότητα κάθε λύσης αξιολογείται μέσω της συνάρτησης:
\[
f = \frac{1}{T},
\]
όπως άλλωστε αναλύσαμε και παραπάνω όπου \(T\) είναι ο συνολικός χρόνος διέλευσης:
\[
T = \sum_{i=1}^{17} \left( t_i + a_i \frac{x_i}{1 - \frac{x_i}{c_i}} \right).
\]
Η συνάρτηση καταλληλότητας μεγιστοποιεί το αντίστροφο του συνολικού χρόνου, προκειμένου οι λύσεις με μικρότερο χρόνο διέλευσης να έχουν μεγαλύτερη πιθανότητα επιλογής.

\section*{\gr Επιλογή Γονέων (Τροχός της Τύχης)}

Η επιλογή των γονέων βασίζεται στη μέθοδο ρουλέτας \en (Roullete Method)\gr. Η πιθανότητα επιλογής μιας λύσης είναι ανάλογη της καταλληλότητάς της:
\[
p_i = \frac{f_i}{\sum_{j} f_j}.
\]
Αυτή η διαδικασία προσομοιώνει τη φυσική επιλογή, δίνοντας προτεραιότητα στις λύσεις με καλύτερες επιδόσεις, ενώ ταυτόχρονα επιτρέπει την επιλογή λιγότερο καλών λύσεων για τη διατήρηση της ποικιλομορφίας.

\section*{\gr Διασταύρωση \en (Crossover)\gr}

Η διασταύρωση επιτρέπει τη δημιουργία νέων λύσεων (απογόνων) μέσω του συνδυασμού δύο γονέων. Οι απόγονοι δημιουργούνται λαμβάνοντας τον μέσο όρο των ροών των γονέων:
\en
\[
x_i^{\text{offspring}} = \frac{x_i^{\text{parent1}} + x_i^{\text{parent2}}}{2}.
\]
\gr
Η διασταύρωση στοχεύει στην εκμετάλλευση των καλών χαρακτηριστικών των γονέων, με την ελπίδα οι απόγονοι να έχουν ακόμα καλύτερη απόδοση. Εάν οι απόγονοι δεν ικανοποιούν τους περιορισμούς, αντικαθίστανται από τους αρχικούς γονείς.

\section*{\gr Μετάλλαξη \en (Mutation)\gr}

Η μετάλλαξη εισάγει τυχαιότητα στον πληθυσμό (δηλαδή εφαρμόζεται σε τυχαία επιλεγμένα βάρη του χρωμοσώματος), μεταβάλλοντας επιλεγμένες ροές \(x_i\) τυχαία. Αυτή η διαδικασία διασφαλίζει την εξερεύνηση νέων περιοχών του χώρου αναζήτησης και μειώνει την πιθανότητα παγίδευσης σε τοπικά ελάχιστα. Εξασφαλίζεται επίσης ότι οι μεταβολές δεν παραβιάζουν τους περιορισμούς του προβλήματος.
Στην αντίστοιχη συνάρτηση όπου υλοποιώ την μέθοδο μετάλλαξης \en(Mutation) \gr θα παρατηρήσετε ότι μεταβάλλονται τα βάρη τυχαία με βάση κανονική κατανομή:
\en
\[
x_i^{\text{new}} = x_i + \text{RAND} \cdot \sigma,
\]
\gr
όπου \(\sigma\) είναι η τυπική απόκλιση της μετάλλαξης.

\section*{\gr Κριτήριο Τερματισμού}

Ο αλγόριθμος τερματίζει όταν η βελτίωση στις τιμές καταλληλότητας μεταξύ των γενεών γίνει αμελητέα, είναι δηλαδή μικρότερη από ένα προκαθορισμένο κατώφλι. Συγκεκριμένα, η αλλαγή υπολογίζεται ως:
\en
\[
\Delta f = \frac{\|f_{\text{new}} - f_{\text{old}}\|}{\|f_{\text{old}}\|}.
\]
\gr
Αν \(\Delta f < \epsilon\), όπου \(\epsilon\) είναι ένα πολύ μικρό κατώφλι π.χ. \(10^{-20}\), ο αλγόριθμος θεωρείται ότι έχει πλέον συγκλίνει.

\section*{\gr Γενική Μεθοδολογία που ακολουθήθηκε}

Ο γενετικός αλγόριθμος λειτουργεί επαναληπτικά με τα εξής στάδια:
\begin{enumerate}
    \item Δημιουργία αρχικού πληθυσμού.
    \item Υπολογισμός καταλληλότητας  (μέσω \en fitness function) \gr για κάθε λύση.
    \item Επιλογή γονέων μέσω της μεθόδου του τροχού της τύχης.
    \item Δημιουργία νέου πληθυσμού μέσω διασταύρωσης και μετάλλαξης.
    \item Έλεγχος τερματισμού και επανάληψη της διαδικασίας αν χρειάζεται.
\end{enumerate}

\section*{\gr Συμπεράσματα και Παρατησήσεις}
Η μέθοδος των γενετικών αλγορίθμων είναι αποτελεσματική για την επίλυση προβλημάτων βελτιστοποίησης δικτύου, καθώς εξισορροπεί την εξερεύνηση νέων λύσεων και την εκμετάλλευση των ήδη καλών λύσεων. Μέσω της εφαρμογής αυτής της μεθοδολογίας, επιτυγχάνεται η ελαχιστοποίηση του συνολικού χρόνου διέλευσης, ικανοποιώντας όλους τους περιορισμούς.
Όπως θα παρατηρήσετε και στο διάγραμμα (για τυχαίους αναπαραγωγίσιμους αριθμούς) είναι εμφανής η αύξηση του \en Fitness \gr όσο αυξάνονται τα \en Generations \gr δηλαδή ο γενετικός αλγόριθμος βρίσκει λύσεις που βελτιώνουν την απόδοση του συστήματος, δηλαδή μειώνουν τον συνολικό χρόνο διέλευσης. Επίσης παρατηρείται ότι το πρόβλημα συγκλίνει σχετικά γρήγορα, κάτι που δείχνει ότι οι περιορισμοί και η φύση της συνάρτησης ελαχιστοποίησης επιτρέπουν την αποτελεσματική εύρεση του χώρου λύσεων.
Η αύξηση της καταλληλότητας δείχνει ότι ο αλγόριθμος είναι αποδοτικός στη βελτίωση της λύσης με κάθε γενιά. Ο γενετικός αλγόριθμος συγκλίνει σε μια λύση που ελαχιστοποιεί τον συνολικό χρόνο διέλευσης του δικτύου, ενώ τηρεί όλους τους περιορισμούς.
\subsection*{\gr Aποτελέσματα και \en Plot}
Ορισμένα αποτελέσματα αλλά και το \en Plot \gr που αναπαριστά την εξέλιξη της καταλληλότητας σε διαδιχικές γενιές αναπαρίστανται παρακάτω.

\begin{figure}[h!]
    \centering
    \includegraphics[width=0.76\textwidth]{StandardV.png}
    \caption{\gr Εξέλιξη Καταλληλότητας σε Διαδοχικές Γενιές.}
\end{figure}

\vspace{0.2cm}
\hspace{-0.6cm}Εχουμε λοιπόν:
\en
\begin{itemize}
    \item Number of generations until convergence is 11
    \item Final Chromosome (NEW POPULATION) is:
[39.54 11.36 24.20 24.90 23.66 15.88 3.43 7.93 14.20 10.70 7.95 6.41 31.97 15.37 35.91 39.03 18.65] 
    \item Total time is 1146.79
    \item PASSED. (\gr Σημαίνει ότι πληρούνται οι περιορισμοί που επιδείχτηκαν στην μαθηματική ανάλυση δοσμένου όμως ενός κατωφλίου  \(10^{-6}\) καθώς προέκυπτε \en floating point precision error \gr και δυσκολία σύγκλισης.)
\end{itemize}
\gr

\hspace{-0.6cm}Ο αντίστοιχος κώδικας με ολόκληρη την υλοποίηση του γενετικού αλγορίθμου σε \(MATLAB\) βρίσκεται στο αρχείο \(Part1.m\).

\chapter{\gr Γενετικός Αλγόριθμος - Μεταβλητός Ρυθμός Εισερχόμενων Οχημάτων}

Στο Θέμα 3 εξετάζεται η περίπτωση όπου ο ρυθμός εισερχόμενων οχημάτων \(V\) μπορεί να μεταβάλλεται έως και κατά \(\pm15\%\) της αρχικής του τιμής \(V_0 = 100\). Η νέα υλοποίηση διατηρεί τη βασική μεθοδολογία του γενετικού αλγορίθμου, με την προσθήκη της δυναμικής τροποποίησης του \(V\).

\section*{\gr Τροποποίηση του Ρυθμού Εισερχόμενων Οχημάτων}

Ο νέος \(V\) υπολογίζεται ως εξής:
\[
V = V_0 \cdot (0.85 + 0.30 \cdot \text{\en RAND \gr}),
\]
όπου \en \(\text{RAND}\) \gr είναι τυχαίος αριθμός στο διάστημα \([0, 1]\). Αυτό διασφαλίζει ότι το \(V\) κυμαίνεται μεταξύ \(0.85V_0\) και \(1.15V_0\).

\section*{\gr Δημιουργία Αρχικού Πληθυσμού \en (Initial Population)\gr}

Η δημιουργία του αρχικού πληθυσμού ακολουθεί την ίδια διαδικασία όπως περιγράφηκε στο Θέμα 2, με τις παραμέτρους να προσαρμόζονται στον νέο ρυθμό εισερχόμενων οχημάτων \(V\).

\section*{\gr Συνάρτηση Καταλληλότητας \en (Fitness Function)\gr}

Η συνάρτηση καταλληλότητας παραμένει ίδια, αλλά ο συνολικός χρόνος διέλευσης υπολογίζεται με βάση τον νέο \(V\). Αυτό διασφαλίζει ότι οι λύσεις αξιολογούνται υπό διαφορετικές συνθήκες ροής, προσδίδοντας ανθεκτικότητα στον αλγόριθμο.

\section*{\gr Εξέλιξη του Πληθυσμού}

Η διαδικασία της επιλογής γονέων, της διασταύρωσης και της μετάλλαξης παραμένει ίδια. Ωστόσο, κάθε γενιά προσαρμόζεται στις νέες συνθήκες του \(V\), κάτι που επιτρέπει στον αλγόριθμο να ανταποκρίνεται σε δυναμικές αλλαγές.

\section*{\gr Συμπεράσματα, Αποτελέσματα και Παρατηρήσεις}
Το διάγραμμα που προκύπτει από την εξέλιξη της καταλληλότητας απεικονίζεται παρακάτω:

\begin{figure}[h!]
    \centering
    \includegraphics[width=0.76\textwidth]{NOTStandardV.png}
    \caption{\gr Εξέλιξη Καταλληλότητας σε Διαδοχικές Γενιές με Δυναμικό Ρυθμό Εισερχόμενων Οχημάτων \(V\).}
\end{figure}

\hspace{-0.6cm}Για μια τυχαία τιμή του \(V\), το αποτέλεσμα του γενετικού αλγορίθμου είναι:
\en
\begin{itemize}
    \item The value of V is now 88.40
    \item Number of generations until convergence is 13
    \item Final Chromosome (NEW POPULATION) is:
[22.60 12.10 20.85 32.85 14.37 8.23 10.67 1.42 20.28 12.57 16.74 21.30 4.52 17.96 5.46 32.33 29.31]
    \item Total time is 763.47
    \item PASSED. (\gr Σημαίνει ότι πληρούνται οι περιορισμοί που επιδείχτηκαν στην μαθηματική ανάλυση δοσμένου όμως ενός κατωφλίου  \(10^{-6}\) καθώς προέκυπτε \en floating point precision error \gr και δυσκολία σύγκλισης.)
\end{itemize}
\gr

\hspace{-0.6cm}Ο γενετικός αλγόριθμος αποδεικνύεται ανθεκτικός στις μεταβολές του \(V\). Οι δυναμικές αλλαγές στον ρυθμό εισερχόμενων οχημάτων δεν επηρεάζουν τη σύγκλιση του αλγορίθμου, ενώ ο συνολικός χρόνος διέλευσης προσαρμόζεται στις νέες συνθήκες. Η προσέγγιση αυτή επιβεβαιώνει την ευελιξία και την αποτελεσματικότητα των γενετικών αλγορίθμων σε προβλήματα δυναμικής βελτιστοποίησης. Επίσης για το συγκεκριμένο \en Run \gr (Παράδειγμα) μπορούμε να παρατηρήσουμε ότι έτυχε το \(V\) να είναι 88.40 δηλαδή μικρότερο από την τιμή 100 που είχε πριν και ο συνολικός χρόνος είναι μικρότερος από ότι πριν όπου το \(V\) ήταν ίσο με 100 πράγμα που είναι και το επιθυμητό καθώς όσο μειώνεται η κυκλοφοριακή συμφόρηση μειώνεται και ο χρόνος. Τέλος αξίζει να σημειωθεί ότι τόσο στο Θέμα 3 όσο και στο Θέμα 2 πολλές φορές ανάλογα με τις τυχαίες μεταβλητές από το \(rand\) σε πολλά \en Runs \gr ενδέχεται να εγκλωβιστούμε σε τοπικό ελάχιστο και να δυσκολέψει η σύγκλιση και να αυξηθεί ο χρόνος ή τα \en iterations\gr. Για την αντιμετώπηση αυτού ίσως χρειάζεται περεταίρω \en fine-tuning \gr ορισμένων παραμέτρων.

\vspace{0.3cm}

\hspace{-0.6cm}Ο αντίστοιχος κώδικας με ολόκληρη την υλοποίηση του γενετικού αλγορίθμου σε \(MATLAB\) βρίσκεται στο αρχείο \(Part2.m\).





\bibliographystyle{plain}
\begin{thebibliography}{3} 
    \bibitem{rovithakis}
    Γεώργιος Α. Ροβιθάκης, \textit{Τεχνικές Βελτιστοποίησης}. Εκδόσεις ΤΖΙΟΛΑ.

    \en
    \bibitem{baeldung}
    https://www.baeldung.com/cs/genetic-algorithms-roulette-selection
    
    \bibitem{mutationStep}
    https://stackoverflow.com/questions/27756477/mutation-step-size-in-genetic-algorithm
\end{thebibliography}


\end{document}
