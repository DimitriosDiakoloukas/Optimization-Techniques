\documentclass[a4paper,12pt]{report}

\usepackage{ucs}
\usepackage[utf8x]{inputenc} % Input encoding for Greek characters
\usepackage[greek,english]{babel} % Language support

\newcommand{\en}{\selectlanguage{english}}
\newcommand{\gr}{\selectlanguage{greek}}

\usepackage{algorithm2e}
\usepackage{algorithm}
\usepackage{algorithmic}
\usepackage{amsmath}
\usepackage{graphicx} % For including images
\usepackage{titlesec} % Custom title formatting
\usepackage{fancyhdr} % For custom headers and footers
\usepackage{geometry} % For adjusting page margins

\geometry{top=2.5cm, bottom=2.5cm, left=2.5cm, right=2.5cm}

\titleformat{\chapter}[display]
    {\normalfont\LARGE\bfseries} % Smaller size and bold for chapter heading
    {\chaptername\ \thechapter} % Chapter number format
    {15pt} % Space between chapter number and title
    {\bfseries} % Smaller size and bold for chapter title
\begin{document}

\begin{titlepage}
    \centering
    \vspace*{-3cm}
    % University logo
    \includegraphics[width=1\textwidth]{auth_logo.png} % Replace with your actual logo file

    % University name in Greek
    \textbf{\gr ΑΡΙΣΤΟΤΕΛΕΙΟ ΠΑΝΕΠΙΣΤΗΜΙΟ ΘΕΣΣΑΛΟΝΙΚΗΣ}
    \vspace{2cm}

    % Document title and subtitle in Greek
    \LARGE\textbf{\gr Τεχνικές Βελτιστοποίησης Αναφορά} \\
    \Large\normalfont{\gr Εργασία 1} \\
    \vspace{4cm}

    \gr
    \large
    \textbf{Διακολουκάς Δημήτριος} \\
    \textbf{AEM 10642}
    \vspace{2.5cm}

    \en
    \textit{Email: ddiakolou@ece.auth.gr}
\end{titlepage}

\gr
\tableofcontents

\chapter{\gr Ορισμός συναρτήσεων (Εισαγωγή)}
Ο στόχος της εργασίας είναι να μειωθούν στο ελάχιστο οι τρεις συγκεκριμένες συναρτήσεις που μας έχουν δωθεί $f_1(x)$, $f_2(x)$ και $f_3(x)$, δεδομένου του αρχικού διαστήματος \([-1,3]\):

\begin{itemize}
    \item \( f_1(x) = (x - 2)^2 + x \cdot \ln(x + 3) \)
    \item \( f_2(x) = e^{-2x} + (x - 2)^2 \)
    \item \( f_3(x) = e^x \cdot (x^3 - 1) + (x - 1) \cdot \sin(x) \)
\end{itemize}

\vspace{1cm}
\hspace{-0.6cm}Οι μέθοδοι που θα χρησιμοποιηθούν διακρίνονται σε δύο κατηγορίες: 

\begin{enumerate}
    \item \textbf{Μέθοδοι χωρίς παραγώγους:}
    \begin{itemize}
        \item Μέθοδος της Διχοτόμου
        \item Μέθοδος του Χρυσού Τομέα
        \item Μέθοδος \en Fibonacci\gr
    \end{itemize}
    
    \item \textbf{Μέθοδοι με χρήση παραγώγων:}
    \begin{itemize}
        \item Μέθοδος της Διχοτόμου με χρήση παραγώγων
    \end{itemize}
\end{enumerate}

\vspace{1cm}
\hspace{-0.65cm}Κάθε μέθοδος θα εφαρμοστεί και θα συγκριθεί με τις άλλες, με στόχο την εξαγωγή συμπερασμάτων σχετικά με την αποδοτικότητα και ακρίβειά τους. Η σύγκριση θα γίνει βάσει του πλήθους επαναλήψεων και της ακρίβειας του αποτελέσματος.

\chapter{Μέθοδος της Διχοτόμου}

Η μέθοδος της διχοτόμου, που χρησιμοποιείται για την εύρεση του ελαχίστου μιας κυρτής συνάρτησης, περιγράφεται στη θεωρία και βασίζεται σε δύο κύριες παραμέτρους: το 
\(\epsilon\) (η απόσταση από το κέντρο) και το \(l\) (το μήκος του διαστήματος αναζήτησης). Οι σχετικοί υπολογισμοί και τα διαγράμματα για τη μέθοδο αυτή βρίσκονται στο αρχείο \en \texttt{Part1.m} \gr.

\vspace{0.3cm}

\hspace{-0.6cm}Αρχικά, εξετάζεται πώς μεταβάλλονται οι υπολογισμοί της κάθε αντικειμενικής συνάρτησης όταν το εύρος αναζήτησης παραμένει σταθερό με \(l = 0.01\), ενώ το \(\epsilon\) λαμβάνει τιμές στο διάστημα \([0.0001, 0.007]\). Τα αποτελέσματα απεικονίζονται στα αντίστοιχα διαγράμματα παρακάτω.

\vspace{0.8cm}
\hspace{-0.6cm}\textbf{Σχόλια για τον Κώδικα} \\
\hspace{-0.2cm}Ο κώδικας περιλαμβάνει την κύρια συνάρτηση \en \texttt{dichotomy\_method}\gr, η οποία υπολογίζει το ελάχιστο για μια δοσμένη συνάρτηση και επιστρέφει τον αριθμό επαναλήψεων και τα όρια του διαστήματος. Στον κώδικα περιέχονται οι συναρτήσεις \( f_1(x) \), \( f_2(x) \) και \( f_3(x) \), που ελαχιστοποιούνται στο διάστημα \([-1, 3]\) όπως αναφέρθηκε και στην εισαγωγή. H υλοποίηση του αλγορίθμου της διχοτόμου αναπαρίσταται στο \en Algorithm 1\gr.


\en
\begin{algorithm}[H]
\caption{\gr Μέθοδος Διχοτόμου για την Εύρεση του Ελαχίστου μιας Συνάρτησης\en}
\KwIn{\gr Μια συνάρτηση \( f \), τα άκρα του διαστήματος \( a \), \( b \), η ακρίβεια \( l \), και η παράμετρος \( \epsilon \)\en}
\KwResult{\gr Ο αριθμός των επαναλήψεων \( k \), οι λίστες \( a_{\text{\en vals\gr}} \) και \( b_{\text{\en vals\gr}} \), και η προσεγγιστική ελάχιστη τιμή\en}

\gr
Αρχικοποίηση: \( k = 0 \), \( a_{\text{\en vals\gr}} = [] \), \( b_{\text{\en vals\gr}} = [] \)
\en

\While{$|b - a| \geq l$}{
    \( k \gets k + 1 \) \\
    \gr
    Υπολογισμός των σημείων:
    \en
    \[
    x_1 = \frac{a + b}{2} - \epsilon, \quad x_2 = \frac{a + b}{2} + \epsilon
    \]

    \eIf{$f(x_1) < f(x_2)$}{
    \gr Θέσε \( b \gets x_2 \) 
    \en
    }{
    \gr Θέσε \( a \gets x_1 \) 
    \en
    }
    \gr
   Πρόσθεσε το \( a \) στο \( a_{\text{\en vals\gr}} \) και το \( b \) στο \( b_{\text{\en vals\gr}} \) 
    \en
}
\[
\text{min} = \frac{a_{\text{\en vals\gr}} + b_{\text{\en vals\gr}}}{2}
\]

\end{algorithm}
\gr

\vspace{0.3cm}

\hspace{-0.6cm}Αρχικά, εξετάζεται πώς η μεταβολή της παραμέτρου \(\epsilon\) επηρεάζει τον αριθμό των υπολογισμών και τα αποτελέσματα φαίνονται στο Σχήμα 2.1. 
\vspace{0.3cm}

\hspace{-0.6cm}Στη συνέχεια, με σταθερό \(\epsilon = 0.001\), μελετάται η επίδραση της αλλαγής του \(l\) στο διάστημα  \([0.001, 0.01]\) στον αριθμό των υπολογισμών όπως αναπαρίσταται στο Σχήμα 2.2. 

\vspace{0.3cm}
\hspace{-0.6cm}Στο τέλος, δημιουργούνται διαγράμματα που δείχνουν την εξέλιξη των ορίων \(a_k\) και \(b_k\) συναρτήσει του δείκτη επαναλήψεων \en k\gr, για διάφορες τιμές του \(l\) και πιο συγκεκριμένα αυτές που ορίστηκαν για τα επόμενα διαγράμματα όπως άλλωστε φαίνεται και στο Σχήμα 2.3 είναι οι \([0.1, 0.05, 0.01, 0.005]\), κατά την εφαρμογή της μεθόδου διχοτόμου για καθεμία από τις συναρτήσεις.
\vspace{0.3cm}

\hspace{-0.6cm}Μέσα από αυτούς τους υπολογισμούς και τα διαγράμματα, αναλύεται ο ρυθμός σύγκλισης της μεθόδου σε συνάρτηση με τις παραμέτρους \(\epsilon\) και \(l\), προσφέροντας έτσι μια σαφή εικόνα για την απόδοση της μεθόδου διχοτόμου.

\vspace{0.8cm}
\hspace{-0.6cm}\textbf{Παρατηρήσεις}
\begin{itemize}
    \item Το διάγραμμα δείχνει πανομοιότυπο και για τις τρεις συναρτήσεις, τόσο για το υποερώτημα 1 όσο και για το 2 (Σχήμα 2.1 και Σχήμα 2.2 αντίστοιχα) κάτι που είναι λογικό, καθώς και στις τρεις περιπτώσεις οι υπολογισμοί της αντικειμενικής συνάρτησης επηρεάζονται από τους ίδιους παράγοντες – συγκεκριμένα το \( l \), το \( \epsilon \), και το εύρος του διαστήματος αναζήτησης.
    \item Στο Σχήμα 2.1 (υποερώτημα 1) παρατηρούμε εύκολα ότι όταν αυξάνεται το \( e \), βλέπουμε ότι ο αριθμός των υπολογισμών της αντικειμενικής συνάρτησης επίσης αυξάνεται. Αυτό είναι σωστό καθώς για να επιτευχθεί σύγκλιση, πρέπει να ισχύει η συνθήκη \( e \leq l/2 \). Με την αύξηση του \( e \), η μέθοδος να σταθεροποιείται στο επιθυμητό εύρος του διαστήματος αναζήτησης.
    \item Στο Σχήμα 2.2 (υποερώτημα 2) παρατηρούμε εύκολα ότι όσο αυξάνεται το \( l \), παρατηρείται μείωση στον αριθμό των υπολογισμών της αντικειμενικής συνάρτησης. Λογικά, αυτό συμβαίνει επειδή, καθώς μειώνουμε τους περιορισμούς, απαιτούνται λιγότερες επαναλήψεις για την ολοκλήρωση της αναζήτησης.
    \item Στο Σχήμα 2.3 (υποερώτημα 3) παρατηρούμε σύγκλιση και των τριών συναρτήσεων \( f_1(x) \), \( f_2(x) \) και \( f_3(x) \) στις τιμές 1.15, 2 και 0.5 αντίστοιχα, ενώ για τις διάφορες τιμές \( l \) τα διαγράμματα συμπίπτουν.
    \item Παρατηρούμε ότι στην μέθοδο της διχοτόμου επαληθεύεται ο τύπος:\[
    \left(\frac{1}{2}\right)^{\frac{n}{2}} \leq \frac{l}{b - a}
    \]

\end{itemize}

\begin{figure}[ht!]
    \centering
    \includegraphics[width=0.9\textwidth]{fig1_1.png} 
    \caption{Μεταβολή Αριθμού Υπολογισμών ως προς ε (για εύρος τιμών \([0.0001, 0.007]\)).}
\end{figure}

\begin{figure}[ht!]
    \centering
    \includegraphics[width=0.9\textwidth]{fig1_2.png} 
    \caption{Μεταβολή Αριθμού Υπολογισμών ως προς \( l \) (για εύρος τιμών \([0.001, 0.01]\)).}
\end{figure}

\begin{figure}[ht!]
    \centering
    \includegraphics[width=0.9\textwidth]{fig1_3.png} 
    \caption{Παρουσίαση σύγκλισης συναρτήσεων για διάφορες τιμές του \( l \) (για εύρος τιμών \([0.1, 0.05, 0.01, 0.005]\)).}
\end{figure}

\chapter{Μέθοδος του Χρυσού Τομέα}
'Οπως ακριβώς και η μέθοδος της διχοτόμου έτσι και η μέθοδος του χρυσού τομέα χρησιμοποιείται για την ανεύρεση του ελαχίστου σε μια κυρτή συνάρτηση. Σε αυτήν τη μέθοδο υπάρχει μία σημαντική παράμετρος, το \( l \), που αντιπροσωπεύει το εύρος του διαστήματος αναζήτησης. Οι υπολογισμοί και τα διαγράμματα για την εφαρμογή αυτής της μεθόδου βρίσκονται στο αρχείο \en \texttt{Part2.m}\gr.

\vspace{0.3cm}

\hspace{-0.6cm}Η ανάλυση εξετάζει την επίδραση της τιμής του \( l \) στους υπολογισμούς της αντικειμενικής συνάρτησης, μεταβάλλοντας το \( l \) σε συγκεκριμένο εύρος τιμών. Σε αυτήν την περίπτωση, το \( l \) λαμβάνει τιμές στο διάστημα \([0.001, 0.01]\). Τα αποτελέσματα παρουσιάζονται στο παρακάτω διάγραμμα.

\vspace{0.8cm}
\hspace{-0.6cm}\textbf{Σχόλια για τον Κώδικα} \\
\hspace{-0.2cm}Ο κώδικας περιλαμβάνει την κύρια συνάρτηση \en \texttt{golden\_section\_method}\gr, η οποία υπολογίζει το ελάχιστο για μια δοσμένη συνάρτηση και επιστρέφει τον αριθμό επαναλήψεων και τα όρια του διαστήματος. Στον κώδικα περιέχονται οι συναρτήσεις \( f_1(x) \), \( f_2(x) \) και \( f_3(x) \), που ελαχιστοποιούνται στο διάστημα \([-1, 3]\) ακριβώς όπως και στην μέθοδο της διχοτόμου. H υλοποίηση του αλγορίθμου του χρυσού τομέα αναπαρίσταται στο \en Algorithm 2\gr.
\en
\begin{algorithm}
\caption{\gr Μέθοδος Χρυσού Τομέα για την Εύρεση του Ελαχίστου μιας Συνάρτησης\en}
\KwIn{\gr Μια συνάρτηση \( f \), τα άκρα του διαστήματος \( a \), \( b \), και η ακρίβεια \( l \)\en}
\KwResult{\gr Ο αριθμός των επαναλήψεων \( k \), οι λίστες \( a_{\text{\en vals\gr}} \) και \( b_{\text{\en vals\gr}} \), και η προσεγγιστική ελάχιστη τιμή\en}

\gr Αρχικοποίηση: \en \( k = 0 \), \( a_{\text{vals}} = [] \), \( b_{\text{vals}} = [] \), \( \gamma = 0.618 \)\\
\gr
Υπολογισμός των αρχικών σημείων:
\en
\[
x_1 = a + (1 - \gamma) \cdot (b - a), \quad x_2 = a + \gamma \cdot (b - a)
\]
\gr
και των τιμών \( f_1 = f(x_1) \) και \( f_2 = f(x_2) \).
\en
\\
\While{\en $|b - a| > l$ \gr}{
    \en \( k \gets k + 1 \) 
    
    \gr Αποθήκευση των τιμών \( a \) και \( b \):
    \en
    \[
    a_{\text{vals}} = [a_{\text{vals}}, a], \quad b_{\text{vals}} = [b_{\text{vals}}, b]
    \]

    \If{$f_1 < f_2$}{
        \gr Θέσε \( b \gets x_2 \), \( x_2 \gets x_1 \), και \( f_2 \gets f_1 \) 
        
        Επανυπολόγισε το \( x_1 \):
        \en
        \[
        x_1 = a + (1 - \gamma) \cdot (b - a), \quad f_1 = f(x_1)
        \]
    }
    \Else{
        \gr Θέσε \( a \gets x_1 \), \( x_1 \gets x_2 \), και \( f_1 \gets f_2 \) 
        
        Επανυπολόγισε το \( x_2 \):
        \en
        \[
        x_2 = a + \gamma \cdot (b - a), \quad f_2 = f(x_2)
        \]
    }
}

\gr Η προσεγγιστική ελάχιστη τιμή είναι:
\en
\[
\text{min} = \frac{a_{\text{\en vals\gr}} + b_{\text{\en vals\gr}}}{2}
\]

\end{algorithm}
\gr

\vspace{0.3cm}

\hspace{-0.6cm}Όπως και στην μέθοδο της διχοτόμου έτσι και εδώ στη μέθοδο του χρυσού τομέα πρόκειται να μελετηθεί η επίδραση της αλλαγής του \(l\) στο διάστημα  \([0.001, 0.01]\) και για τις 3 αντικειμενικές συναρτήσεις στον αριθμό των υπολογισμών όπως αναπαρίσταται στο Σχήμα 3.1. 

\vspace{0.3cm}
\hspace{-0.6cm}Επιπλέον και εδώ ζητάται να δημιουργηθούν τα διαγράμματα που δείχνουν την εξέλιξη των ορίων \(a_k\) και \(b_k\) συναρτήσει του δείκτη επαναλήψεων \en k\gr, για διάφορες τιμές του \(l\) και πιο συγκεκριμένα αυτές που ορίστηκαν για τα επόμενα διαγράμματα όπως άλλωστε φαίνεται και στο Σχήμα 3.2 είναι οι \([0.1, 0.05, 0.01, 0.005]\), κατά την εφαρμογή της μεθόδου του χρυσού τομέα για καθεμία από τις συναρτήσεις.
\vspace{0.3cm}

\hspace{-0.6cm}Και σε αυτή την περίπτωση από τους υπολογισμούς και τα διαγράμματα, αναλύεται ο ρυθμός σύγκλισης της μεθόδου σε συνάρτηση με τις παραμέτρους \(l\), προσφέροντας έτσι μια σαφή εικόνα για την απόδοση της μεθόδου χρυσού τομέα.
\newpage
\vspace{0.8cm}
\hspace{-0.6cm}\textbf{Παρατηρήσεις}
\begin{itemize}
    \item Το διάγραμμα φαίνεται να είναι ίδιο και για τις τρεις συναρτήσεις, τόσο για το υποερώτημα 1 (Σχήμα 3.1) κάτι που είναι λογικό, καθώς και στις τρεις περιπτώσεις οι υπολογισμοί της αντικειμενικής συνάρτησης επηρεάζονται από τους ίδιους παράγοντες, συγκεκριμένα το \( l \).
    \item Στο Σχήμα 3.1 (υποερώτημα 1) παρατηρούμε εύκολα ότι όσο αυξάνεται το \( l \), παρατηρείται μείωση στον αριθμό των υπολογισμών της αντικειμενικής συνάρτησης. Λογικά, αυτό συμβαίνει επειδή, καθώς μειώνουμε τους περιορισμούς, απαιτούνται λιγότερες επαναλήψεις για την ολοκλήρωση της αναζήτησης.
    \item Στο Σχήμα 3.2 (υποερώτημα 2) παρατηρούμε σύγκλιση και των τριών συναρτήσεων \( f_1(x) \), \( f_2(x) \) και \( f_3(x) \) στις τιμές 1.15, 2 και 0.5 αντίστοιχα, ενώ για τις διάφορες τιμές \( l \) τα διαγράμματα συμπίπτουν.
    \item Και σε αυτή την περίπτωση φαίνεται να επαληθεύεται ο τύπος: \[
    0.618^{(n-1)} \leq \frac{l}{b - a}
    \]
\end{itemize}

\begin{figure}[ht!]
    \centering
    \includegraphics[width=0.9\textwidth]{fig2_1.png} 
    \caption{Μεταβολή Αριθμού Υπολογισμών ως προς \( l \) (για εύρος τιμών \([0.001, 0.01]\)).}
\end{figure}

\begin{figure}[ht!]
    \centering
    \includegraphics[width=0.9\textwidth]{fig2_2.png} 
    \caption{Παρουσίαση σύγκλισης συναρτήσεων για διάφορες τιμές του \( l \) (για εύρος τιμών \([0.1, 0.05, 0.01, 0.005]\)).}
\end{figure}

\chapter{\grΜέθοδος \en Fibonacci}

Σε αυτό το θέμα καλούμαστε να επαναλάβουμε το θέμα της μεθόδου του χρυσού τομέα αλλά αυτή την φορά χρησιμοποιώντας την μέθοδο \en Fibonacci \gr. Σε αυτήν την περίπτωση πάλι έχουμε μία παράμετρο \( l \) δηλαδή το εύρος του διαστήματος αναζήτησης. Οι υπολογισμοί και τα διαγράμματα για την μέθοδο αυτή βρίσκονται στο αρχείο \en \texttt{Part3.m}\gr. Σε αυτήν την περίπτωση για μία ακόμη φορά θα αποφασίσουμε το \( l \) να λαμβάνει τιμές στο διάστημα \([0.001, 0.01]\). H υλοποίηση του αλγορίθμου της μεθόδου του \en Fibonacci 
 \gr αναπαρίσταται στο \en Algorithm 3\gr.
 
\vspace{0.8cm}
\hspace{-0.6cm}\textbf{Σχόλια για τον Κώδικα} \\
\hspace{-0.6cm}Όπως και στις προηγούμενες μεθόδους έτσι και εδώ στη μέθοδο του \en Fibonacci \gr πρόκειται να μελετηθεί η επίδραση της αλλαγής του \(l\) στο διάστημα  \([0.001, 0.01]\) και για τις 3 αντικειμενικές συναρτήσεις στον αριθμό των υπολογισμών όπως αναπαρίσταται στο Σχήμα 4.1. Ωστόσο εδώ πρέπει νε βρούμε το αριθμό υπολογισμών για την ακολουθία \en Fibonacci \gr Ν αξιοποιώντας το \(l\) αλλά και τα \(a\) και \(b\). Για να βρεθεί, λοιπόν, ο αριθμός αυτός των \en iterations \gr, θα πρέπει πρώτα να αξιοποιήσουμε την συνθήκη της μεθόδου: \[
F_n \geq \frac{b - a}{l}
\]


\vspace{0.3cm}
\hspace{-0.6cm}Επίσης όπως και στο θέμα 2, ζητάται να δημιουργηθούν τα διαγράμματα που δείχνουν την εξέλιξη των ορίων \(a_k\) και \(b_k\) συναρτήσει του δείκτη επαναλήψεων \en k\gr, για διάφορες τιμές του \(l\) και πιο συγκεκριμένα αυτές που ορίστηκαν για τα επόμενα διαγράμματα όπως άλλωστε φαίνεται και στο Σχήμα 4.2 είναι οι \([0.1, 0.05, 0.01, 0.005]\), κατά την εφαρμογή της μεθόδου \en Fibonacci \gr για καθεμία από τις συναρτήσεις.
\vspace{0.3cm}

\hspace{-0.6cm}Και σε αυτή την περίπτωση από τους υπολογισμούς και τα διαγράμματα, αναλύεται ο ρυθμός σύγκλισης της μεθόδου σε συνάρτηση με τις παραμέτρους \(l\) μέσω των οποίων υπολογίζω τον αριθμό υπολογισμών Ν, προσφέροντας έτσι μια σαφή εικόνα για την απόδοση της μεθόδου \en Fibonacci \gr.
\newpage
\vspace{0.8cm}
\hspace{-0.6cm}\textbf{Παρατηρήσεις}
\begin{itemize}
    \item Το διάγραμμα φαίνεται να είναι ίδιο και για τις τρεις συναρτήσεις, τόσο για το υποερώτημα 1 (Σχήμα 4.1) κάτι που είναι λογικό, καθώς και στις τρεις περιπτώσεις οι υπολογισμοί της αντικειμενικής συνάρτησης επηρεάζονται από τους ίδιους παράγοντες, συγκεκριμένα το \( l \).
    \item Στο Σχήμα 4.1 (υποερώτημα 1) παρατηρούμε εύκολα ότι όσο αυξάνεται το \( l \), παρατηρείται μείωση στον αριθμό των υπολογισμών της αντικειμενικής συνάρτησης. Λογικά, αυτό συμβαίνει επειδή, καθώς μειώνουμε τους περιορισμούς, απαιτούνται λιγότερες επαναλήψεις για την ολοκλήρωση της αναζήτησης.
    \item Στο Σχήμα 4.2 (υποερώτημα 2) παρατηρούμε σύγκλιση και των τριών συναρτήσεων \( f_1(x) \), \( f_2(x) \) και \( f_3(x) \) στις τιμές 1.15, 2 και 0.5 αντίστοιχα, ενώ για τις διάφορες τιμές \( l \) τα διαγράμματα συμπίπτουν.
\end{itemize}

\vspace{-0.3cm}
\en
\begin{algorithm}
\caption{\gr Μέθοδος \en Fibonacci \gr για την Εύρεση του Ελαχίστου μιας Συνάρτησης\en}
\KwIn{\gr Μια συνάρτηση \( f \), τα άκρα του διαστήματος \( a \), \( b \), η ακρίβεια \( \epsilon \), και ο αριθμός των βημάτων \( N \)\en}
\KwResult{\gr Ο αριθμός των επαναλήψεων \( k \), οι λίστες \( a_{\text{\en vals\gr}} \), \( b_{\text{\en vals\gr}} \) και \( \text{min\_vals} \), και η προσεγγιστική ελάχιστη τιμή\en}

\gr Υπολογισμός της ακολουθίας \en Fibonacci \gr με \( N \) όρους:
\en
\[
\text{fib} = \text{fibonacci\_sequence}(N)
\]

\gr Αρχικοποίηση: \en \( k = 0 \), \( a_{\text{vals}} = [] \), \( b_{\text{vals}} = [] \), \( \text{min} = [] \)\\
\gr
Υπολογισμός των αρχικών σημείων:
\en
\[
x_1 = a + \frac{\text{fib}(N-2)}{\text{fib}(N)} \cdot (b - a), \quad x_2 = a + \frac{\text{fib}(N-1)}{\text{fib}(N)} \cdot (b - a)
\]
\gr
και των τιμών \( f_1 = f(x_1) \) και \( f_2 = f(x_2) \).
\en
\\
\For{\en $i = 1$ \gr \textbf{to} \en $N - 1$ \gr}{
    \en
    \[
    a_{\text{vals}} = [a_{\text{vals}}, a], \quad b_{\text{vals}} = [b_{\text{vals}}, b], \quad \text{min} = [\text{min}, \frac{a + b}{2}]
    \]
    \If{\en $i \neq N - 2$ \gr}{
        \If{$f_1 < f_2$}{
            \gr Θέσε \( b \gets x_2 \), \( x_2 \gets x_1 \), και \( f_2 \gets f_1 \) 
            
            Επανυπολόγισε το \( x_1 \):
            \en
            \[
            x_1 = a + \frac{\text{fib}(N - i - 2)}{\text{fib}(N - i)} \cdot (b - a), \quad f_1 = f(x_1)
            \]
        }
        \Else{
            \gr Θέσε \( a \gets x_1 \), \( x_1 \gets x_2 \), και \( f_1 \gets f_2 \) 
            
            Επανυπολόγισε το \( x_2 \):
            \en
            \[
            x_2 = a + \frac{\text{fib}(N - i - 1)}{\text{fib}(N - i)} \cdot (b - a), \quad f_2 = f(x_2)
            \]
        }
    }
    \Else{
        \gr Θέσε \( x_2 \gets x_1 + \epsilon \) και \( f_2 = f(x_2) \)\en
        
        \If{$f_1 < f_2$}{
            \gr Θέσε \( b \gets x_2 \) \en
        }
        \Else{
            \gr Θέσε \( a \gets x_1 \) \en
        }
        
        \en
        \[
        a_{\text{vals}} = [a_{\text{vals}}, a], \quad b_{\text{vals}} = [b_{\text{vals}}, b], \quad \text{min} = [\text{min}, \frac{a + b}{2}]
        \]
        \gr Αύξησε το \( k \) κατά 1 και τερμάτισε το βρόχο.
        \en
        \textbf{break} 
    }
    \newpage
    \en \( k \gets k + 1 \) 
}
\end{algorithm}

\gr
\newpage
\begin{figure}[ht!]
    \centering
    \includegraphics[width=0.9\textwidth]{fig3_1.png} 
    \caption{Μεταβολή Αριθμού Υπολογισμών ως προς \( l \) (για εύρος τιμών \([0.001, 0.01]\)).}
\end{figure}

\begin{figure}[ht!]
    \centering
    \includegraphics[width=0.9\textwidth]{fig3_2.png} 
    \caption{Παρουσίαση σύγκλισης συναρτήσεων για διάφορες τιμές του \( l \) (για εύρος τιμών \([0.1, 0.05, 0.01, 0.005]\)).}
\end{figure}



\chapter{\gr Μέθοδος της Διχοτόμου με παράγωγο}
Η μέθοδος της διχοτόμου με χρήση παραγώγου στοχεύει στην εύρεση του ελαχίστου μιας συνάρτησης μέσω της παραγώγου της. Παρακάτω επισυνάπτεται και ο αλγόριθμος που αναπαρίσταται στο \en Algorithm 4\gr. Οι υπολογισμοί και τα διαγράμματα για την μέθοδο αυτή βρίσκονται στο αρχείο \en \texttt{Part4.m}\gr.

\vspace{0.8cm}
\hspace{-0.6cm}\textbf{Σχόλια για τον Κώδικα} \\
\hspace{-0.6cm}Όπως και στην μέθοδο \en Fibonacci \gr, έτσι και σε αυτήν την περίπτωση πάλι έχουμε μία παράμετρο \( l \) δηλαδή το εύρος του διαστήματος αναζήτησης. Σε αυτήν την περίπτωση για μία ακόμη φορά θα αποφασίσουμε το \( l \) να λαμβάνει τιμές στο διάστημα \([0.001, 0.01]\), ώστε να μελετηθεί η επίδραση της αλλαγής του και για τις 3 αντικειμενικές συναρτήσεις στον αριθμό των υπολογισμών όπως αναπαρίσταται στο Σχήμα 5.1.
Πάλι και στο θέμα αυτό θα πρέπει βασιζόμενοι στις τιμές των \(l\), \(a\) και \(b\) να βρούμε τον αριθμό υπολογισμών Ν. Για να βρεθεί, λοιπόν, ο αριθμός αυτός των \en iterations \gr, θα πρέπει πρώτα να αξιοποιήσουμε την συνθήκη της μεθόδου: \[
\left(\frac{1}{2}\right)^{\frac{n}{2}} \leq \frac{l}{b - a}
\]

\vspace{0.3cm}

\hspace{-0.6cm}Επίσης όπως και στο θέμα 2, ζητάται να δημιουργηθούν τα διαγράμματα που δείχνουν την εξέλιξη των ορίων \(a_k\) και \(b_k\) συναρτήσει του δείκτη επαναλήψεων \en k\gr, για διάφορες τιμές του \(l\) και πιο συγκεκριμένα αυτές που ορίστηκαν για τα επόμενα διαγράμματα όπως άλλωστε φαίνεται και στο Σχήμα 5.2 είναι οι \([0.1, 0.05, 0.01, 0.005]\), κατά την εφαρμογή της μεθόδου της Διχοτόμου με χρήση παραγώγων για καθεμία από τις συναρτήσεις.
\vspace{0.3cm}

\hspace{-0.6cm}Και σε αυτή την περίπτωση από τους υπολογισμούς και τα διαγράμματα, αναλύεται ο ρυθμός σύγκλισης της μεθόδου σε συνάρτηση με τις παραμέτρους \(l\) μέσω των οποίων υπολογίζω τον αριθμό υπολογισμών Ν, προσφέροντας έτσι μια σαφή εικόνα για την απόδοση της μεθόδου της Διχοτόμου με χρήση παραγώγων.

\vspace{0.8cm}
\hspace{-0.6cm}\textbf{Παρατηρήσεις}
\begin{itemize}
    \item Το διάγραμμα φαίνεται να είναι ίδιο και για τις τρεις συναρτήσεις, τόσο για το υποερώτημα 1 (Σχήμα 5.1) κάτι που είναι λογικό, καθώς και στις τρεις περιπτώσεις οι υπολογισμοί της αντικειμενικής συνάρτησης επηρεάζονται από τους ίδιους παράγοντες, συγκεκριμένα το \( l \).
    \item Στο Σχήμα 5.1 (υποερώτημα 1) παρατηρούμε εύκολα ότι όσο αυξάνεται το \( l \), παρατηρείται μείωση στον αριθμό των υπολογισμών της αντικειμενικής συνάρτησης. Λογικά, αυτό συμβαίνει επειδή, καθώς μειώνουμε τους περιορισμούς, απαιτούνται λιγότερες επαναλήψεις για την ολοκλήρωση της αναζήτησης.
    \item Στο Σχήμα 5.2 (υποερώτημα 2) παρατηρούμε σύγκλιση και των τριών συναρτήσεων \( f_1(x) \), \( f_2(x) \) και \( f_3(x) \) στις τιμές 1.15, 2 και 0.5 αντίστοιχα, ενώ για τις διάφορες τιμές \( l \) τα διαγράμματα συμπίπτουν.
\end{itemize}
\vspace{0.8cm}

\en
\begin{algorithm}
\caption{\gr Μέθοδος της Διχοτόμου με Παράγωγο για την Εύρεση του Ελαχίστου μιας Συνάρτησης\en}
\KwIn{\gr Η παράγωγος μιας συνάρτησης \( df \), τα άκρα του διαστήματος \( a \), \( b \), και ο μέγιστος αριθμός επαναλήψεων \( N \)\en}
\KwResult{\gr Ο αριθμός των επαναλήψεων \( k \), οι λίστες \( a_{\text{\en vals\gr}} \) και \( b_{\text{\en vals\gr}} \), και η προσεγγιστική ελάχιστη τιμή\en}

\gr Αρχικοποίηση: \en \( k = 1 \), \( a_{\text{vals}} = [a] \), \( b_{\text{vals}} = [b] \)\\

\While{\en $k < N$ \gr}{
    \gr Υπολογισμός του σημείου \( x_k = \frac{a + b}{2} \) και της παραγώγου \( df(x_k) \).
    \[
    x_k = \frac{a + b}{2}, \quad df_{x_k} = df(x_k)
    \]
    \en
    \If{\en $df_{x_k} == 0$ \gr}{
        \gr Διακοπή του αλγορίθμου
        \en
        \textbf{break} 
    }
    \en
    \If{\en $df_{x_k} > 0$ \gr}{
        \gr Θέσε \( b \gets x_k \) 
    }
    \en
    \Else{
        \gr Θέσε \( a \gets x_k \) 
    }

    \gr Ενημέρωση των λιστών:
    \en
    \[
    a_{\text{vals}} = [a_{\text{vals}}, a], \quad b_{\text{vals}} = [b_{\text{vals}}, b]
    \]

    \gr Αύξησε τον δείκτη επανάληψης \( k \gets k + 1 \).
}
\end{algorithm}

\gr

\begin{figure}[ht!]
    \centering
    \includegraphics[width=0.9\textwidth]{fig4_1.png} 
    \caption{Μεταβολή Αριθμού Υπολογισμών ως προς \( l \) (για εύρος τιμών \([0.001, 0.01]\)).}
\end{figure}

\begin{figure}[ht!]
    \centering
    \includegraphics[width=0.9\textwidth]{fig4_2.png} 
    \caption{Παρουσίαση σύγκλισης συναρτήσεων για διάφορες τιμές του \( l \) (για εύρος τιμών \([0.1, 0.05, 0.01, 0.005]\)).}
\end{figure}

\chapter{\gr Αποτελέσματα Συγκρίσεων}
Είναι εμφανές ότι η πειραματική αξιολόγηση της αποδοτικότητας των μεθόδων χωρίς χρήση παραγώγου (διχοτόμου, χρυσού τομέα και \en Fibonacci \gr) συμφωνεί με τη θεωρητική κατάταξη τους. Συγκεκριμένα, από την περισσότερο προς την λιγότερο αποδοτική μέθοδο, η σειρά είναι η εξής:
\begin{enumerate}
    \item Μέθοδος \en Fibonacci \gr
    \item Μέθοδος Χρυσού Τομέα
    \item Μέθοδος Διχοτόμου
\end{enumerate}
Ωστόσο η μέθοδος Διχοτόμου με παράγωγο απαιτεί τον μικρότερο αριθμό υπολογισμών για να επιτύχει το ελάχιστο σε όλο το εύρος τιμών \( l \) και έτσι είναι η πιο αποδοτική από όλες.

\vspace{0.3cm}

\hspace{-0.6cm}Επιπλέον, καθώς το \( k \) αυξάνεται (δηλαδή όταν το \( l \) μειώνεται), παρατηρούμε ότι η και οι 4 μέθοδοι συγκλίνουν στα ίδια σημεία για κάθε συνάρτηση.


\bibliographystyle{plain}
\begin{thebibliography}{1}
    \bibitem{rovithakis}
    Γεώργιος Α. Ροβιθάκης, \textit{Τεχνικές Βελτιστοποίησης}. Εκδόσεις ΤΖΙΟΛΑ.
\end{thebibliography}

\end{document}
